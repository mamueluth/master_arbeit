%% Version: 0.8 (12.02.2019)

%% My_document_info.tex
%% Copyright 2020 KIT, IAR-IPR (Denis Štogl)
%% Mail: denis.stogl@kit.edu
%
% This work may be distributed and/or modified under the
% conditions of the LaTeX Project Public License, either version 1.3
% of this license or (at your option) any later version.
% The latest version of this license is in
%   http://www.latex-project.org/lppl.txt
% and version 1.3 or later is part of all distributions of LaTeX
% version 2005/12/01 or later.
%
% This work has the LPPL maintenance status `maintained'.
%
% The Current Maintainer of this work is M. Y. Name.
%
% This work consists of the files pig.dtx and pig.ins
% and the derived file thesis.tex.


\title{Distributed Control of Multiple Robots Using ros2\_control}

\titleotherlanguage{Verteilte Mehrmaschinen-Regelung einer Roboterzelle mittels ros2\_control
}

\author{Manuel Muth}
\address{Berckmüllerstraße 1C}
\city{76131 Karlsruhe}
\email{uqdut@student.kit.edu}

\keywords{distributed control, multiple controllers, ros2\_control, ROS2, control}
\keywordsotherlanguge{verteilte Regelung, mehrere Regler, ros2\_control, ROS2, Regelung, Regelungstechnik}

%% Study program or a seminar/subject
\studyprogram{Intelligente Industrieroboter}

%% IMPORTANT: Seminar only: If not "Seminar Intelligente Industrieroboter" uncomment this
% \nogrouplogo

%% Name of your institute (Default: IAR-IPR)
% \institute{Test}
%% Name of your faculty (Default: KIT-Fakultät für Informatik
% \KITfaculty{This is my Faculty}
%% Address of your institute (Default: Engler-Bunte-Ring 8)
% \instituteaddress{}
% %% Insitute City (Default: 76131 Karlsruhe)
% \institutecity{}

\reviewerone{Prof. Dr.-Ing. habil. Björn Hein}
\reviewertwo{Prof. Dr.-Ing. habil. Thomas Längle}
%
% %% The advisors are PhDs or Postdocs
\advisorone{Dr.-Ing. Denis \v{S}togl}
% %% The second advisor can be omitted
\advisortwo{}
%
% %% Please enter the start end end time of your thesis (for techreport not needed)
\editingtime{03. Januar 2023}{03. Juli 2023}

%% --------------------------------
%% | Settings for word separation |
%% --------------------------------
% Help for separation:
% In german package the following hints are additionally available:
% "- = Additional separation
% "| = Suppress ligation and possible separation (e.g. Schaf"|fell)
% "~ = Hyphenation without separation (e.g. bergauf und "~ab)
% "= = Hyphenation with separation before and after
% "" = Separation without a hyphenation (e.g. und/""oder)

% Describe separation hints here:
\hyphenation{
% Pro-to-koll-in-stan-zen
% Ma-na-ge-ment  Netz-werk-ele-men-ten
% Netz-werk Netz-werk-re-ser-vie-rung
% Netz-werk-adap-ter Fein-ju-stier-ung
% Da-ten-strom-spe-zi-fi-ka-tion Pa-ket-rumpf
% Kon-troll-in-stanz
}

%%
%% --------------------
%% |   Bibliography   |
%% --------------------
\newcommand{\mybibliographyfiles}{Bibliography/ipr_articles,Bibliography/kit_template_example_bibliography,Bibliography/masterthesis}


%% --------------------
%% |     Acronyms     |
%% --------------------
\newacronym{dof}{DOF}{Degree of Freedom}
\newacronym{dds}{DDS}{OMG Data-Distribution Service}
\newacronym{ipr}{IAR-IPR}{Institute for Anthropomatics and Robotics - Intelligent Process Control and Robotics}
\newacronym{kit}{KIT}{Karlsruhe Institute of Technology}
\newacronym{qos}{QoS}{Quality of Service}
\newacronym{ptp}{PTP}{Precision Time Protocol}
\newacronym{rcl}{rcl}{\gls{ros} Client Library}
\newacronym{rclcpp}{rclcpp}{\gls{ros} Client Library C++}
\newacronym{rclpy}{rclpy}{\gls{ros} Client Library Python}
\newacronym{rmw}{rmw}{\gls{ros} middleware}
\newacronym{ros}{ROS}{Robot Operating System}
\newacronym{ros2}{ROS 2}{Robot Operating System 2}
\newacronym{rsi}{RSI}{RobotSensorInterface}
%% --------------------
%% |     Glossary     |
%% --------------------
\newglossaryentry{ci}
{
    name={\textit{CommandInterface}},
    description={}
}
\newglossaryentry{ddsg}
{
    name={\gls{dds}},
    description={\acrlong{dds} publish-subscribe a communication based middleware standard for real-time and embedded systems by \cite{pardo-castellote_omg_2003, schlesselman_omg_2004, noauthor_data_nodate}}
}
\newglossaryentry{dofg}
{
    name={\gls{dof}},
    description={\acrlong{dof}}
}
\newglossaryentry{node}
{
    name={\textit{node}},
    description={Can be understood as a fundamental software component that is used for communication and computation in \gls{ros} and \gls{ros2}. Builds the organizing endpoint int the computional graph. It can perform various tasks such as sensing, actuating, processing, and interfacing with other nodes to exchange data or trigger actions},
    plural={\textit{nodes}}
}
\newglossaryentry{nodelet}
{
    name={\textit{nodelet}},
    description={Simliar to a \gls{node} in \gls{ros} but with the ability to run multiple nodlets in one process},
    plural={\textit{nodelets}}
}
\newglossaryentry{message}
{
    name={\textit{message}},
    description={Datatype in \gls{ros} and \gls{ros2} used when communication via \glspl{topic}},
    plural={\textit{messages}}
}
\newglossaryentry{rc}
{
    name={\textit{ros\_control}},
    description={Framework for robot control in \gls{ros}}
}
\newglossaryentry{ri}
{
    name={\textit{ReferenceInterface}},
    description={}
}
\newglossaryentry{rosg}
{
    name={\gls{ros}},
    description={The \acrlong{ros} is a middleware based on a publisher-subsciber model. \gls{rosg} is mainly used in robotics and provides services for hardware abstraction and low-level control. Further, it comprises tools, libraries, and functionalities used in robotic applications}
}
\newglossaryentry{ros2g}
{
    name={\gls{ros2}},
    description={\acrlong{ros2} is the second generation of the \acrlong{ros} \cite{macenski_robot_2022}}
}
\newglossaryentry{r2c}
{
    name={\textit{ros2\_control}},
    description={Framework for real-time capable control using \gls{ros2}}
}
\newglossaryentry{rsig}
{
    name={\gls{rsi}},
    description={The \acrlong{rsi} serves as an interface for communication between the industrial robot and its sensor system }
}
\newglossaryentry{service}
{
    name={\textit{service}},
    description={A service is a communication mechanism in \gls{ros} and \gls{ros2} that allows \glspl{node} to request and receive a response in a client-server pattern},
    plural={\textit{services}}
}
\newglossaryentry{si}
{
    name={\textit{StateInterface}},
    description={}
}
\newglossaryentry{sm}
{
    name={shared-memory},
    description={Shared-memory architecture is a concept that allows multiple processors or threads to access a common physical memory. Data can then be exchanged over the common memory. }
}
\newglossaryentry{topic}
{
    name={\textit{topic}},
    description={In the context of \gls{ros} and \gls{ros2} a topic is a communication mechanism. Topics provide a publisher-subscriber based exchange mechanism of \glspl{message} for \glspl{node}},
    plural={\textit{topics}}
}