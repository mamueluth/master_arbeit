\Abstract{Industrial robots are often controlled by the robot controllers distributed by the manufacture. Those robot controllers appear to the user as monolithic black boxes. However, manufacturing plants are required to be more flexible and respond to fast customer demands. \gls{ros2} with the framework \textit{ros2\_control} for robot control provide the basis for flexible control of robots. The \textit{ros2\_control} framework is built around a shared-memory architecture and aims at controlling a single robot or small robot cell. In this work, a concept for distributed control over the network with the \textit{ros2\_control} framework is presented. The proposed concept introduces a central controller manager with multiple sub-controller manages, which allows different control architectures. The concept allows drivers and controllers to run on different computers, allowing one controller to control multiple robots over the network. Further, controllers running on different machines can be chained, allowing for distributed control of multiple robots. The proposed concept has been tested with a test setup of three different computers and two industrial robots of the type KUKA KR3 R540. The impact of different combinations of ROS 2 middlewares and Quality of Service policy on the control quality has been investigated. The evaluation found the chosen middleware to influence the quality of the control result. The results indicate that a Quality of Service policy with reliability set to BEST\_EFFORT should be preferred over one set to RELIABLE to achieve better performance.}
