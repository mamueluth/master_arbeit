%% ==============================
\chapter{\iflanguage{ngerman}{Stand der Wissenschaft und Technik}{State of the Art}}\todo{Add 2-3 examples with 2-4 sentences.}
\label{sec:state_of_the_art}

Networked control systems are a current trend in industrial automation and are applied in a variety of different areas such as manufacturing systems, automobiles and unmanned aerial vehicles. Key research directions in this area are "control of networks" and "control over networks". Control of networks aims at providing good \glspl{qos} in order to provide a satisfactory performance over the network. The focuses of control over network is on addressing the constraints introduced by the network and to enhance the robustness of the control approach \cite{zhang_networked_2020}.
% In a networked control system, sensor data and control signals are transmitted from the sensor to controller and controller to actuator via a communication channels. This transmission is subject to delays or even package loss. The two most common models to deal with such network induced issues are the random model and the input delay model \cite{zhang_networked_2020}.

% \section{Distributed Control}
% \paragraph{Example 1}
% \paragraph{Cloud}

\section{Control Over Network in \gls{ros2}}\label{c2_sec_control_over_network}
Gutiérrez et al. showed in their paper that with \gls{ros2} version Ardent, initially released in December 2017, it became feasible to do soft real-time robotics over the network \cite{gutierrez_towards_2018}. The study evaluated the communication capabilities of \gls{ros2} in a scenario
\begin{wrapfigure}{r}{7cm}
\includegraphics[width=7cm]{Figures/c2/Gutiérrez_toward_distributed_and_real-time.png}
\caption{The experimental setup used by Gutiérrez et al. to evaluate the real-time performance of \gls{ros2} communication over Ethernet. Figure taken from \cite{gutierrez_towards_2018}.} \label{c2_fig_gutierrez_experimental_seutp}
\end{wrapfigure}
for communication between robot components under Linux. By measuring worst-case latencies and missed deadlines, the authors analyzed the impact of computation and network congestion on communication latencies. The experimental setup can be seen in figure \ref{c2_fig_gutierrez_experimental_seutp} and consisted of an embedded device and a Linux PC. They then measured the round-trip time between the embedded device and the Linux PC. This was done with different \gls{dds} middleware implementations. With their results, they showed that configuring the \gls{ros2} framework and \gls{dds}  threads significantly reduced jitter and worst-case latencies. However, in their work they also highlighted the limitations of the Linux network stack, especially for non-critical traffic. Using Linux traffic control \gls{qos} methods, those limitations could be minimized. They concluded that soft real-time communication over Ethernet is possible. The results also show that hard real-time is not possible.\cite{gutierrez_towards_2018}. \newline
In further investigations on the subject, Puck et al. concluded  that the performance could be improved when using time-synchronization systems. In their research, they used the \gls{ptp} to synchronize time between the systems. However, the Linux network stack continued to introduce non-deterministic latencies and jitter \cite{puck_distributed_2020}. They were then even able to show that, under certain conditions, hard real-time requirements can be met up to frequencies of 1kHz. This was only achievable with a certain system configuration \cite{puck_performance_2021}.\newline
Plasberg et al. conducted their research on a distributed and time-synchronized setup consisting of two real-time capable Linux computers. They connected the Linux PCs via Ethernet for communication. They also used a separate network for the time-synchronization with \gls{ptp}. For this, they needed two network interfaces on each system and a PTP-capable switch with boundary clock, that was connected to a GPS-based grandmaster. They then conducted cyclic latency tests. As a result, Plasberg et al. concluded their setup to be capable of controlling robots in a distributed system with soft real-time requirements, like industrial robot arms or mobile platforms \cite{plasberg_towards_2022}.
% \section{Distributed Control of Multiple Robots}

% \todoin{Old has to be removed}
% Enabling QoS for Collaborative Robotics Applications with Wireless TSN \cite{sudhakaran_enabling_2021}\newline\newline

% Keine Ahnung lauter bla bla bla das in die cloud verlegt werden soll der neue große trend aber weiß nicht ob das so wirklich passt "Edge robotics: are we ready? an experimental evaluation of current vision and future directions" \cite{groshev_edge_2023} \newline
% Hier dass distributed network control das trending Forschunthema ist aber glaub das ist nicht die Richtung die ich mache: Networked Control Systems: A Survey of Trends and Techniques \cite{zhang_networked_2020} \newline

