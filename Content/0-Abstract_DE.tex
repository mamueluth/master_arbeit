\Abstract{Industrieroboter werden häufig von Robotersteuerungen, die vom Hersteller vertrieben werden, gesteuert. Diese Robotersteuerungen erscheinen dem Benutzer dabei als monolithische Blackboxen. In der heutigen Zeit müssen Fertigungsanlagen jedoch flexible sein und schnell auf Kundenwünsche reagieren. ROS 2 mit dem Framework \textit{ros2\_control} für die Robotersteuerung bietet die Grundlage für eine flexible Regelung von Robotern. Das \textit{ros2\_control}-Framework basiert auf einer Shared-Memory-Architektur und zielt auf die Regelung eines einzelnen Roboters oder einer kleinen Roboterzelle ab. In dieser Arbeit wird ein Konzept zur verteilten Regelung über das Netzwerk unter Verwendung des \textit{ros2\_control}-Frameworks vorgestellt. Das vorgeschlagene Konzept beinhaltet einen zentralen Controller-Manager mit mehreren Sub-Controller-Managern. Dies ermöglicht verschiedene Regelungsarchitekturen. Das Konzept erlaubt es z.b. Treiber und Regler auf verschiedenen Rechnern laufen zu lassen, sodass ein Computer mehrere Roboter über das Netzwerk regeln kann. Darüber hinaus können Regler, die auf verschiedenen Rechnern laufen, miteinander verkettet werden, was eine verteilte Regelung mehrerer Roboter ermöglicht. Das vorgeschlagene Konzept wurde mit einem Testaufbau aus drei verschiedenen Computern und zwei Industrieroboter vom Typen KUKA KR3 R540 getestet. Es wurde untersucht, wie sich verschiedene Kombinationen von ROS 2 Middlewares und Quality-of-Service-Parametern auf die Regelungsqualität auswirken. Die Auswertung zeigte, dass die gewählte Middleware die Qualität des Regelungsergebnisses beeinflusst. Die Ergebnisse zeigen weiterhin, dass die Quality-of-Service-Parameter mit einer Zuverlässigkeit von BEST\_EFFORT solchen mit RELIABLE vorzuziehen sind, um bessere Ergebnisse zu erzielen.}