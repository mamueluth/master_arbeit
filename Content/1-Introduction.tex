%% ==============================
% Part is used only in PhD thesis
\part{The challenges}
\chapter{\iflanguage{ngerman}{Einleitung}{Introduction}}
\label{sec:Introduction}
%% ==============================
\section{Motivation}
\todoin{check this again some useful additions to motivation:\cite{bilancia_overview_2023}}
Robots are playing a more and more vital role in every sector of our lives. This can be seen especially in the industrial, automotive and electronic field, where they have been increasingly used in the last 30 years \cite{cheng_rise_2019, bilancia_overview_2023}. However, robotic systems are also increasingly appearing in our everyday lives, whether through robotic vacuum cleaners or robotic mowers. The basis of all these robotic systems are the drivers and controllers that enable the control and thus the movement of these robotic systems. However, today many of the controllers for robotic control are monolithic black-boxes built by the robot manufacturers. Those black-boxes are neither accessible nor exchangeable. This is often due to the fact of them finely tuned to fulfill the task of robot control for the robots distributed by the manufacture \cite{puck_distributed_2020, plasberg_towards_2022}. \newline
With the appearance of \gls{ros} in 2007, an open-source framework for robotics development, and packages such as \gls{rc}, it is possible to combine those monolithic controllers with simply interfaces. This allows the creation of complex robotic applications. By 2009 \gls{ros} had established itself as 'de facto' standard for robotics software development. However, \gls{ros} has its limitations which lead to the development of second version. The second version, called \gls{ros2}, was built from the ground up with the goal of addressing those limitations. \newline
With \gls{ros2} the \gls{r2c} framework for real-time capable control emerged. The \gls{r2c} framework is a rewrite of \gls{rc} with the goal to simplify integration of new hardware and overcome drawbacks present in \gls{rc}. However, the current version of \gls{r2c} is focused on controlling a single robot or small robot cell and is based on a \gls{sm} architecture. This does not make use of the utilized middleware \gls{dds} with its full support for message exchange over the network. 

\section{Scope of this Thesis}
This thesis concentrates solely on \gls{ros2} and especially on the question of the feasibility to realize a distributed control concept over the network in \gls{r2c}. It aims to evaluate how such a concept could be integrated into the existing \gls{r2c} framework. Further, the goal is to test if the proposed concept is capable of controlling multiple robots. Moreover, is aims at evaluating what influence the chosen middleware of \gls{ros2} has. Respectively, chosen parameters and quality of service settings of the \gls{rmw} are compared and evaluated.


\section{Thesis Outline}
\paragraph{\hyperref[sec:state_of_the_art]{Section 2: State of the Art}}

\paragraph{\hyperref[sec:background]{Section 3: Background}}
In the third chapter, an overview and introduction to frameworks and technologies this thesis relies on is given. First, \gls{ros} as well as its successor \gls{ros2} are introduced. Then a quick introduction to the \acrlong{dds} is given and a quick overview of different middlewares for \gls{ros2} are presented. Thereafter, the current state of the \gls{r2c} framework is presented and the most important concepts for this thesis are introduced, like hardware abstraction and chaining of controllers. Further, it is shown how in the current version of \gls{r2c}, based on \acrlong{sm}, multiple robots can be controlled and what the drawbacks and limitations are.
\paragraph{\hyperref[sec:concept]{Section 4: Concept}}
In the fourth chapter, the concepts that underlie this work are outlined. It is presented how this thesis is build on top of concepts from \gls{ros2} like \glspl{topic} and \glspl{service}. Further, the idea of how to integrate everything in the existing framework of \gls{r2c}.

\paragraph{\hyperref[sec:implementation]{Section 5: Implementation}}
In this chapter, the actual realization of the concept presented in chapter 4 is delineated. 
\paragraph{\hyperref[sec:results]{Section 6: Results}}
In chapter six, the results of the experimental evaluation are shown. In the first part, the experimental setup is described. It is explained on what hardware the experiments were conducted. This includes the computers for running the controllers as well as the used robot platform. The network and which \gls{dds} implementations are used is described as well. In addition, different parameters for \gls{dds} are used and compared. \newline
Then the actual results are presented and discussed.\todo{Maybe small summery?}

\paragraph{\hyperref[sec:conclusion_and_outlook]{Section 7: Conclusion and Outlook}}
In the last chapter of this thesis, the presented concept as well as the actual implementation are discussed and based on the results are set into context.\todo{Maybe small summery?}

% \todoin{Old has to be removed}
% Roboter immer wichtiger, überall Roboter. \gls{ros2} de facto Standard. Steuerung und Regelung von Roboter ist Grundlage von all den Anwendungen, was nützt Roboter, wenn er sich nicht bewegt. Das \gls{r2c} Framework ist für Steuerung und Regelung in \gls{ros2} das Standard-Framework. Nur \gls{sm} Architektur. \todoAddin{Warum wollen wir verteilte Regelung?}